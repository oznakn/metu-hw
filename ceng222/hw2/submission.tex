\documentclass[12pt]{article}
\usepackage[utf8]{inputenc}
\usepackage{float}
\usepackage{amsmath}
\usepackage{amssymb}

\usepackage[hmargin=3cm,vmargin=6.0cm]{geometry}
\topmargin=-2cm
\addtolength{\textheight}{6.5cm}
\addtolength{\textwidth}{2.0cm}
\setlength{\oddsidemargin}{0.0cm}
\setlength{\evensidemargin}{0.0cm}
\usepackage{indentfirst}
\usepackage{amsfonts}

\begin{document}

\section*{Student Information}

Name : Ozan Akın \\

ID : 2309599 \\


\section*{Answer 1}
\subsection*{a)}
    \begin{flalign*}
        \indent E(X) &= \sum\limits_{x} x \cdot fP_X(x) & \\
        \indent &= 0 \cdot P_X(0) + 1 \cdot P_X(1) + 2 \cdot P_X(2) & \\
        \indent &= 0 \cdot \dfrac{3}{12} + 1 \cdot \dfrac{6}{12} + 2 \cdot \dfrac{3}{12} & \\
        \indent &= 0 + \dfrac{1}{2} + \dfrac{1}{2} & \\
        \indent &= 1
    \end{flalign*}{}
    
    \begin{flalign*}
        \indent Var(X) &= E(X^2) - \mu^2 & \\
        \indent &= E(X^2) - 1^2 & \\
        \indent &= 0^2 \cdot P_X(0) + 1^2 \cdot P_X(1) + 2^2 \cdot P_X(2) - 1 & \\
        \indent &= 0 \cdot \dfrac{3}{12} + 1 \cdot \dfrac{6}{12} + 4 \cdot \dfrac{3}{12} - 1 & \\
        \indent &= \dfrac{1}{2} + 1 - 1 & \\
        \indent &= \dfrac{1}{2} &
    \end{flalign*}{}

\subsection*{b)}
    Let $Z$ be $X + Y$.
    
    \begin{flalign*}
        \indent P_Z(0) &= P(0, 0) = \dfrac{1}{12} & \\
        \indent P_Z(1) &= P(1, 0) = \dfrac{4}{12} & \\
        \indent P_Z(2) &= P(0, 2) + P(2, 0) = \dfrac{2}{12} + \dfrac{1}{12} = \dfrac{3}{12} = \dfrac{1}{4}  & \\
        \indent P_Z(3) &= P(1, 2) = \dfrac{2}{12} = \dfrac{1}{6} & \\
        \indent P_Z(4) &= P(2, 2) = \dfrac{2}{12} = \dfrac{1}{6} &
    \end{flalign*}{}

\subsection*{c)}
    $Cov(X, Y) = E(XY) - E(X)E(Y)$
    
     \begin{flalign*}
        \indent E(X) &= \sum\limits_{x} x \cdot fP_X(x) & \\
        \indent &= 0 \cdot P_X(0) + 1 \cdot P_X(1) + 2 \cdot P_X(2) & \\
        \indent &= 0 \cdot \dfrac{3}{12} + 1 \cdot \dfrac{6}{12} + 2 \cdot \dfrac{3}{12} & \\
        \indent &= 0 + \dfrac{1}{2} + \dfrac{1}{2} & \\
        \indent &= 1 &
    \end{flalign*}{}
    
     \begin{flalign*}
        \indent E(Y) &= \sum\limits_{y} y \cdot fP_Y(y) & \\
        \indent &= 0 \cdot P_Y(0) + 2 \cdot P_Y(2) & \\
        \indent &= 0 \cdot \dfrac{6}{12} + 2 \cdot \dfrac{6}{12} & \\
        \indent &= 1 &
    \end{flalign*}{}
    
    \begin{flalign*}
        \indent E(XY) &= \sum\limits_{x}\sum\limits_{y} (xy) \cdot P(x, y) & \\
        \indent &= 0 \cdot 0 \cdot \dfrac{1}{12} + 1 \cdot 0 \cdot \dfrac{4}{12} + 2 \cdot 0 \cdot \dfrac{1}{12} & \\
        \indent & \; \; \; + 0 \cdot 2 \cdot \dfrac{2}{12} + 1 \cdot 2 \cdot \dfrac{2}{12} + 2 \cdot 2 \cdot \dfrac{2}{12} & \\
        \indent &= \dfrac{4}{12} + \dfrac{8}{12} = \dfrac{12}{12} & \\
        \indent &= 1 &
    \end{flalign*}{}
 
    \begin{flalign*}
        \indent Cov(X, Y) &= E(XY) - E(X)E(Y) & \\
        \indent Cov(X, Y) &= 1 - 1 \cdot 1 & \\
        \indent &= 0 &
    \end{flalign*}{}
    
\subsection*{d)}
    Recall that $P(a, b) = P_A(a) \cdot P_B(b)$ if $A$ and $B$ are independent.
    
    \begin{flalign*}
        \indent E(AB) &= \sum\limits_{a}\sum\limits_{b} (ab) \cdot P(a, b) & \\
        \indent &= \sum\limits_{a}\sum\limits_{b} a \cdot P_A(a) \cdot b \cdot P_B(b) & \\
        \indent &= \sum\limits_{a} a \cdot P_A(a) \sum\limits_{b} b \cdot P_B(b) & \\
        \indent &= E(A)E(B) &
    \end{flalign*}{}
    
    \indent Since $Cov(A, B) = E(AB) - E(A)E(B)$,  $Cov(A, B) = 0$.

\subsection*{e)}
     If $A$ and $B$ are independent, then $P(x, y) = P_X(x) \cdot P_Y(y)$.
     
     \begin{flalign*}
         \indent P_Y(0) &= \dfrac{1}{12} + \dfrac{4}{12} + \dfrac{1}{12} = \dfrac{1}{2} & \\
         \indent P_X(0) &= \dfrac{1}{12} + \dfrac{2}{12} = \dfrac{1}{4} & \\
         \indent P(0, 0) &= \dfrac{1}{12} \neq P_X(0) \cdot P_Y(0) = \dfrac{1}{8} &
     \end{flalign*}{}
     
     \indent As a result $X$ and $Y$ are dependent.


\section*{Answer 2}
\subsection*{a)}
    We can calculate the result subtracting the probability of no broken pen, 1 broken pen, or 2 broken pen from 1. Let $P(x)$ be the function of the probability of $x$ broken pen.
    
    \begin{flalign*}
        \indent P(0) &= (0.8)^{12} \cong 0.06871947 & \\
        \indent P(1) &= (0.8)^{11} \cdot 0.2 \cdot C(12, 1) \cong 0.20615843 & \\
        \indent P(2) &= (0.8)^{10} \cdot (0.2)^2 \cdot C(12, 2) \cong 0.2834678415 &
    \end{flalign*}{}
    
    \indent $1 - P(0) - P(1) - P(2) \cong 0.441654251$.

\subsection*{b)}
    There are 12 different pens. In order to be fifth pen we test second broken pen, the one of the four pens must be broken. \\
    
    \indent Thus the probability is $(0.8)^3 \cdot 0.2 \cdot C(4, 1) \cdot 0.2 \cong 0.08192$.

\subsection*{c)}
    The probability of broken pen is $0.2$. That means for every $5$ pen, $1$ pen is broken. Since we want to find $4$ broken pens, we have to look $5 \cdot 4 = 20$ pen in average, to find $4$ broken pens.

\section*{Answer 3}
\subsection*{a)}
    If Bob gets a phone call every 4 hours, that means Bob gets a $0.25$ phone call every hour. We can use Exponential Distribution to compute the probability.
    
    \begin{flalign*}
        \indent P\{T \geq 2\} &= 1 - F(2) = 1 - 1 + e^{- \lambda \cdot 2} & \\
        \indent &= e^{- \frac{1}{2}} & \\
        \indent & \cong 0.6065306597 &
    \end{flalign*}{}

\subsection*{b)}
    If Bob gets a phone call every 4 hours, that means Bob gets a $0.25$ phone call every hour, $2.5$ phone call every 10 hours. We can use Poisson Distribution to compute the probability. Let X be the number of calls, and X has Poisson Distribution with parameter $\lambda = 2.5$. From table A3 from the textbook,
    
    \begin{flalign*}
        \indent & P\{X \leq 3\} = F_X(3) \cong 0.758 & \\
    \end{flalign*}{}
    
\subsection*{c)}
    If Bob gets a phone call every 4 hours, that means Bob gets a $0.25$ phone call every hour, $2.5$ phone call every 10 hours, $4$ phone call every 16 hours. We can use Poisson Distribution to compute the probability. \\
    
    Let X be the number of calls, and X has Poisson Distribution with parameter $\lambda = 2.5$. Let Y be the number of calls, and Y has Poisson Distribution with parameter $\lambda = 1.5$. Using the table A3 from the textbook,
    
    \begin{flalign*}
        \indent & P\{X > 3\} = 1 - F_X(3) = 1 - 0.758 = 0.242 & \\
        \indent & P\{Y > 0\} = 1 - F_Y(0) = 1 - 0.223 = 0.777
    \end{flalign*}{}
    \begin{flalign*}
        \indent & P\{X > 2\} = 1 - F_X(3) = 1 - 0.544 = 0.456 & \\
        \indent & P\{Y > 1\} = 1 - F_Y(0) = 1 - 0.558 = 0.442
    \end{flalign*}{}
    \begin{flalign*}
        \indent & P\{X > 1\} = 1 - F_X(3) = 1 - 0.287 = 0.713 & \\
        \indent & P\{Y > 2\} = 1 - F_Y(0) = 1 - 0.809 = 0.191
    \end{flalign*}{}
    \begin{flalign*}
        \indent & P\{X > 0\} = 1 - F_X(3) = 1 - 0.082 = 0.918 & \\
        \indent & P\{Y > 3\} = 1 - F_Y(0) = 1 - 0.934 = 0.066
    \end{flalign*}{}
    
    $0.242 \cdot 0.777 + 0.456 \cdot 0.442 + 0.713 \cdot 0.191 + 0.918 \cdot 0.066 \cong 0.586357$
    
\end{document}

