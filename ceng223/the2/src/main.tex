\documentclass[11pt]{article}
\usepackage[utf8]{inputenc}
\usepackage{float}
\usepackage{amsmath}
\usepackage{amssymb}
\usepackage[shortlabels]{enumitem}

\usepackage[hmargin=3cm,vmargin=6.0cm]{geometry}
%\topmargin=0cm
\topmargin=-2cm
\addtolength{\textheight}{6.5cm}
\addtolength{\textwidth}{2.0cm}
%\setlength{\leftmargin}{-5cm}
\setlength{\oddsidemargin}{0.0cm}
\setlength{\evensidemargin}{0.0cm}

\begin{document}

\section*{Student Information } 
%Write your full name and id number between the colon and newline
%Put one empty space character after colon and before newline
Full Name : Ozan Akın \\
Id Number : 2309599 \\

% Write your answers below the section tags
\section*{Answer 1}

\begin{enumerate}[a)]
    \item
        \begin{enumerate}[(i)]
            \item $A \cap (B \cup C)$
            \item $(A \cap B) \cup C$
            \item $A - ((A \cap B) - C)$
        \end{enumerate}{}
        ~ \\
    \item 
        \begin{enumerate}[(i)]
            \item $(A \times B) \times C  = A \times (B \times C)$
                \begin{itemize}
                    \item Let A, B, C different sets given as; \\
                        $A = \{a, b\}$ \\
                        $B = \{c, d\}$ \\
                        $C = \{e\}$
                    \item $A \times B = \{(a, c), (a, d), (b, c), (b, d)\}$ \\
                        $B \times C = \{(c, e), (c, e)\}$
                    \item $(A \times B) \times C = \{((a, c), e), ((a, d), e), ((b, c), e), ((b, d), e)\}$ \\
                        $A \times (B \times C) = \{(a, (c, e)), (a, (d, e)), (b, (c, e)), (b, (d, e))\} $
                \end{itemize}{}
            As a result, $(A \times B) \times C \neq A \times (B \times C)$ \\
            
            \item ~ 
\begin{table}[H]
    \centering
    $(A \cap B) \cap C = A \cap (B \cap C)$ \\
    ~ \\
    \begin{tabular}{c|c|c|c|c|c|c}
         $A$ & $B$ & $C$ & $A \cap B$ & $B \cap C$ & $(A \cap B) \cap C$ & $A \cap (B \cap C)$ \\
         \hline
         $1$ & $1$ & $1$ & $1$ & $1$ & $1$ & $1$ \\
         $1$ & $1$ & $0$ & $1$ & $0$ & $0$ & $0$ \\
         $1$ & $0$ & $1$ & $0$ & $0$ & $0$ & $0$ \\
         $1$ & $0$ & $0$ & $0$ & $0$ & $0$ & $0$ \\
         $0$ & $1$ & $1$ & $0$ & $1$ & $0$ & $0$ \\
         $0$ & $1$ & $0$ & $0$ & $0$ & $0$ & $0$ \\
         $0$ & $0$ & $1$ & $0$ & $0$ & $0$ & $0$ \\
         $0$ & $0$ & $0$ & $0$ & $0$ & $0$ & $0$ \\
         
    \end{tabular}

    As a result, $(A \cap B) \cap C = A \cap (B \cap C)$. \\
\end{table}


\pagebreak

            \item ~
\begin{table}[H]
    \centering
    $(A \oplus B) \oplus C = A \oplus (B \oplus C)$ \\
    ~ \\
    \begin{tabular}{c|c|c|c|c|c|c}
         $A$ & $B$ & $C$ & $A \oplus B$ & $B \oplus C$ & $(A \oplus B) \oplus C$ & $A \oplus (B \oplus C)$ \\
         \hline
         $1$ & $1$ & $1$ & $0$ & $0$ & $1$ & $1$ \\
         $1$ & $1$ & $0$ & $0$ & $1$ & $0$ & $0$ \\
         $1$ & $0$ & $1$ & $1$ & $1$ & $0$ & $0$ \\
         $1$ & $0$ & $0$ & $1$ & $0$ & $1$ & $1$ \\
         $0$ & $1$ & $1$ & $1$ & $0$ & $0$ & $0$ \\
         $0$ & $1$ & $0$ & $1$ & $1$ & $1$ & $1$ \\
         $0$ & $0$ & $1$ & $0$ & $1$ & $1$ & $1$ \\
         $0$ & $0$ & $0$ & $0$ & $0$ & $0$ & $0$ \\
         
    \end{tabular}

    As a result, $(A \oplus B) \oplus C = A \oplus (B \oplus C)$. \\
\end{table}
        \end{enumerate}{}
\end{enumerate}{}

\hfill

\section*{Answer 2}

\begin{enumerate}[a)]
    \item 
        $A_0 \subseteq f^{-1}(f(A_0))$
        
        \begin{itemize}
            \item To show $A_0 \subseteq f^{-1}(f(A_0))$, we have to prove that $\forall x(x \in A_0 \rightarrow x \in f^{-1}(f(A_0))$.
            \item Since $x \in A_0$ and $A_0 \subseteq A$, there must be $f(x) \in B$.
            \item Since $f$ is injective, $f(a) = f(b) \rightarrow a = b$, there must be $C_{0} \subseteq B$ such that $f(x) \in C_{0}$.
            \item As a result, there must be $x \in f^{-1}(C_{0})$.
            \item Since also $f(A_0) = C_{0}$, $A_0 = f^{-1}(f(A_0))$
            \item This also implies $A_0 \subseteq f^{-1}(f(A_0))$
        \end{itemize}{}
        
    \item
        $f(f^{-1}(B_0)) \subseteq B_0$
        
        \begin{itemize}
            \item To show $f(f^{-1}(B_0)) \subseteq B_0$, we have to show that $\forall x (x \in f(f^{-1}(B_0)) \rightarrow x \in B_0)$.
            \item Let $f^{-1}(B_0) \in C_0$, then $f(C_0) = B_0$.
            \item $f(C_0) \subseteq B_0$ if and only if $f$ is surjective, since there can be elements that does not belong to $f(f^{-1}(B_0))$.
            \item As a result, $f(f^{-1}(B_0)) \subseteq B_0$.
        \end{itemize}{}
\end{enumerate}{}

\pagebreak

\section*{Answer 3}

Let $A$ be a nonempty set. Show that the following are equivalent

\begin{enumerate}[(i)]
    \item A is countable
    \item There is a surjective function $f : \mathbb{Z}^{+} \rightarrow A$
    \item There is a injective function $f : A \rightarrow  \mathbb{Z}^{+} $
\end{enumerate}{}

\hfill

\begin{enumerate}[1)]
    \item If $A$ is countable, that means either $A$ is a finite set, or $A$ has same cardinality with the set $Z^+$ (defn. 3 from textbook p. 171).
    \item The sets $A$ and $\mathbb{Z}^{+}$ have the same cardinality if and only if there is one to one correspondence from $A$ to $\mathbb{Z}^{+}$ (defn. 1 from textbook p. 170).
    \item If there is one to one correspondence from $A$ to $\mathbb{Z}^{+}$, then they the cardinality of $A$ is less then or same as the cardinality of $\mathbb{Z}^{+}$ (defn. 2 from textbook p. 170).
    \item Let $f : \mathbb{Z}^{+} \rightarrow A$. If $A$ is countable and $|A| \leq |\mathbb{Z}^{+}|$, then using (3) we can define $f$ such that f is a surjective function. $(i) \rightarrow (ii)$
    \item Let $g : A \rightarrow \mathbb{Z}^{+}$. As a result from $f$ and (3) we can define $g$ as injective, since there is one to one correspondence from $A$ to $\mathbb{Z}^{+}$. $(ii) \rightarrow (iii)$
    \item Since $|A| \leq |\mathbb{Z}^{+}|$, we can say that $A$ is countable using (1). $(iii) \rightarrow (i)$
\end{enumerate}{}

\hfill

\section*{Answer 4}

\begin{enumerate}[a)]
    \item
        Show that the set of finite binary strings is countable.
        
        \begin{itemize}
            \item We can define a function $f$ from binary strings which is a sequence of $0s$ and $1s$ to positive integers such as $f("0001") = 1$, $f("0101") = 5$ etc.
            \item Since we can enumerate positive integers, $\mathbb{Z}^{+}$ is countable, the set of finite binary strings must be countable.
        \end{itemize}{}
    \item
        Show that the set of infinite binary strings is uncountable.
        \begin{itemize}
            \item If the consider infinite binary strings, then we cannot define a function $f$ from binary strings which is a sequence of $0s$ and $1s$ to positive integers since there will be infinite positive integers that can be represented by a binary string.
            \item Since we cannot define the function $f$, we cannot represent binary strings as positive integers.
            \item Since we cannot enumerate the infinite binary strings as they are uncountable, also the set of infinite binary string is uncountable.
        \end{itemize}{}
\end{enumerate}{}

\pagebreak

\section*{Answer 5}

\begin{enumerate}[a)]
    \item Determine whether $\log n!$ is $\Theta (n \log n)$.
        \begin{enumerate}[1)]
            \item Consider that $\log (a \cdot b) = \log (a) + \log (b)$.
            \item In order to prove $\log n!$ is $\Theta (n \log n)$, we have to prove both $\log n!$ is $O (n \log n)$ and $n \log n$ is $\Omega (\log n!)$.
            \item $\log (n!) = \log (n \cdot (n-1) \cdot (n-2) \cdot (n-3) \dots 1)$ \\
                $\log (n!) = \log (n) + \log (n-1) + \log (n-2) + \dots + \log (1)$ \\
                $\log (n!) \leq \log n + \log n + \log n + \dots + \log n$ (n times) \\
                $\log (n!) \leq n \log n$
            \item So, $\log n!$ is $O (n \log n)$.
            \item $\log (n!) = \log (n \cdot (n-1) \cdot (n-2) \cdot (n-3) \dots 1)$ \\
                $\log (n!) = \log (n) + \log (n-1) + \log (n-2) + \dots + \log (1)$ \\
                $\log (n!) \geq \log (n) + \dots + \log (\frac{n}{2}+1) + \log (\frac{n}{2})$  (delete second half) \\
                $\log (n!) \geq \log (\frac{n}{2}) + \dots + \log (\frac{n}{2}) + \log (\frac{n}{2})$ (replace all by $\frac{n}{2}$, which is much smaller) \\
                $\log (n!) \geq \frac{n}{2} \cdot \log (\frac{n}{2})$
            \item So, $\log n!$ is $\Omega (n \log n)$.
            \item As a result from both (4) and (6), $\log n!$ is $\Theta (n \log n)$.
        \end{enumerate}{}
        
    \item Which function grows faster, $n!$ or $2^n$?
        \begin{itemize}
            \item Let $f(n) = \dfrac{n!}{2^n}$
            \item $L = \lim_{n \rightarrow \infty} |\dfrac{f(n+1)}{f(n)}|$
                $ = \lim_{n \rightarrow \infty} |\dfrac{(n+1)!}{2^{n+1}} \cdot \dfrac{2^n}{n!}|$
                $ = \lim_{n \rightarrow \infty} |\dfrac{(n+1) \cdot n!}{2 \cdot 2^n} \cdot \dfrac{2^n}{n!}|$
                $ = \dfrac{n+1}{2}$
            \item For $n \in \mathbb{Z}^+, L \geq 1$.
            \item As a result, from ratio test $n!$ grows faster than $2^n$.
        \end{itemize}{}
\end{enumerate}



\end{document}
